\chapter{Présentation de la mission}
\thispagestyle{fancy}

\section{l'intranet GRA2NET}

les collaborateurs de la DSI de Groupama Rhône-Alpes Auvergne utilisent au
quotidien GRA2NET. Cet intranet rassemble de nombreux outils de
gestion et de nombreux liens vers des applications externes. Il est
organisé en onglet regroupant les applications et les liens par corps
de metier (administration réseau, gestion du materiel, téléphonie...).

Parmis les outils les plus utilisés, on retrouve par exemple une liste
des sous-réseaux de l'entreprise, un outils d'installation de
logiciel, un autre permettant de lister les logiciels utilisés et les
statistiques les concernants.


\section{Contexte de la refonte de l'intranet}

L’intranet gra2net a été développé il y a une vingtaine d’année dans
le but de rassembler au même endroit les différents outils de la DSI,
et de permettre un accès rapide aux applications externes. Avec le
temps, de nouveaux outils sont apparus et gra2net s’est étoffé, mais
de par sa conception et malgré une première refonte en 2002, celui-ci
est difficilement maintenable. Si de nouveaux liens ont été ajoutés,
les liens obsolètes n’ont pas été supprimés ou mis à jour, il s’en est
suivi une navigation de plus en plus compliquée et un désintéressement
progressif de l’intranet.

Cependant, plusieurs outils sont encore utilisés par de nombreux
collaborateurs, c’est pourquoi une refonte de gra2net a été envisagée,
pour d’une part le mettre à jour et l’adapter aux nouveaux usages des
collaborateurs, et d’autre part revoir sa conception pour le rendre
plus souple et permettre ainsi une maintenance facile.


\section{Description de la mission}

La mission qui m'a été donné par Patrick HINSCHBERGER était de faire
d'abord la conception, puis un prototype de la nouvelle version de
gra2net en suivant la méthode de projet recemment mise en place a
Groupama. J'ai donc eu la chance de pouvoir participer à la majorité 
des étapes du projet sous la direction de M Hinschberger. 

\chapter{Conception de la nouvelle version}

\section{Analyse de l'existant}

Pour pouvoir organiser la refonte de l'intranet, il me fallait d'abord
connaitre l'intranet existant. J'ai donc commencé par passer du temps
a prendre l'outil en main, et à le tester afin de pouvoir mettre en
évidence certains des problèmes dont il pouvait souffir.  j'ai ensuite
interviewé tous les utilisateurs de cet intranet (soit la plupart des
membres de la DSI) pour apprendre quelle utilisation de l'intranet ils
avaient, quelles fonctionnalités étaient devenues obsolètes
et quelles fonctionnalités ils souhaitaient voir apparaitre.

Cette tache, même si elle était necessaire, a été assez longue et m'a
pris trois semaines. La principale difficulté fut de pouvoir voir un
maximum de monde car un certain nombre de collaborateur étaient déjà
en vacances ou étaient pris par leur propre projet. Une autre part de
mon temps à également été consacré à la synthèse de chaque interview,
dans le but de pouvoir d'abord créer une macro-expression du besoin,
résumant les axes d'amélioration devant être suivis. J'ai enfin rédigé
le dossier de specification fonctionnelle, beaucoup plus exhaustif sur
les modifications apportés à l'intranet.

L'ancienne version de l'intranet gra2net avait été développé en Visual
Basic et était optimisé pour Internet Explorer 6. Elle exploite des
bases de données gérées par SQL Server 2008 R2, et est organisée en
differents onglets:
\begin{itemize}
\item Gestion
\item Système
\item Exploitation
\item Reseau
\item Statistiques
\item Sécurité
\item Téléphonie\\
\end{itemize}

Comme je l'ai indiqué en introduction, ces onglets regroupent
plusieurs et/ou liens dont un certain nombres sont
obsolètes. L'interface souffre de gros problèmes d'érgonomie, par
exemple, on ne peux faire un clique gauche pour acceder au contenus
d'un onglet, il faut ouvrir un nouvel onglet. Par ailleurs certains
outils possède une partie de leurs modules dans onglet et une partie
dans l'autre (c'est notamment le cas de l'outil de télé-distribution
RADIA, dont l'installeur de de logiciel se trouvait dans la partie
système et dont les modules de statistique se trouvait dans la partie
gestion).

\begin{figure}[h]
\includegraphics[scale=0.55]{Images/gra2net_old.png}
\caption{Ancienne version de l'intranet}
\label{Ancienne version de l'intranet}
\end{figure}

\section{Spécifications fonctionnelles}

Le dossier de spécification fonctionnelle est disponible en annexe, en
voici néanmoins un résumé.

\noindent la conception nouvelle version de l'intranet s'appuie sur trois axes majeurs:
\begin{itemize}
\item une refonte de l'architecture du site, pour accelerer l'accès aux liens/outils.
\item un nettoyage du contenu en supprimant ou en mettant à jour les liens ou les outils obsolètes.
\item La création de nouveaux outils répondant à des besoins de la DSI.\\
\end{itemize} 

\subsection*{Une nouvelle architecture}

\begin{figure}[h]
\begin{center}
\includegraphics[scale=0.55]{Images/Archi_detaillee.png}
\caption{La nouvelle architecture de l'intranet}
\label{La nouvelle architecture de l'intranet}
\end{center}
\end{figure}

La nouvelle architecture du site est prévue pour que tous les liens ou
outils soient accessibles en trois actions au maximum. J'ai décidé
pour cela de fusionner les onglets Systèmes, Gestion et Exploitation
en un seul nommé Système/Gestion et de sortir les fonctions de
télé-distributions dans un onglet dédié, l'idée étant de maintenir une
arborescence qui ne soit pas trop large et avec des onglets plus
cohérents.


\begin{figure}[h]
\begin{center}
\includegraphics[scale=0.80]{Images/menu.png}
\caption{Nouvelle version de l'intranet}
\label{Nouvelle version de l'intranet}
\end{center}
\end{figure}


\subsection*{Création de nouveaux outils}

Avec le temps de nouveaux besoins sont apparus et il est ressortis de mes interviews
que de nouveaux outils devaient être créés:
\begin{description}
\item[Outil gestion de serveur :] Cet outil doit permettre aux
  collaborateurs de gérer les stocks de serveur (par exemple savoir
  sur quel site se trouve un serveur, quels sont les serveurs non
  utilisé etc.).
\item[Outil gestion de documentation :] Cet doit permettre de
  regrouper toute la documentation au même endroit.
\item[Outil extranet :] Comme pour la documentation l'idée est de
  regrouper les différents extranets de chaque onglet au sein d'un même outil.
\end{description}



\chapter{réalisation}

Le dossier de specification technique est disponible en annexe, mais
nous allons en résumer le principal ici.  Les choix techniques ont été
faits selon plusieurs critères:

\begin{itemize}
\item Faciliter la maintenance.
\item Ameliorer la fiabilité des outils.
\end{itemize}

Le langage choisi à été le PHP, combiné à un framework interne à
Groupama. Le developpement s'est fait en utlisant le pattern MVC
(modèle-view-controller). L'idée était d'utiliser les outils connus et
maitrisé des équipes de la DSI de Groupama afin de facilité la
maintenance. Les bases de données existante étaient gérées par SQL
server 2008 RC, j'ai donc continué à l'utiliser. L'intranet était
hébergé par un serveur administré par windows server 2008 utilisant
IIS (Internet Information Service). Enfin j'ai utilisé eclipse comme IDE 
de développement et j'ai utilisé les moteurs d'internet explorer 9 et de 
chrome pour tester le rendu des pages au fur et à mesure.


\section{sécurité de l'intranet}

L'intranet doit être accessible à tous les collaborateurs de la DSI,
et les informations qu'il contient rapidement accessible, cependant
les bases de données alimentées via les interfaces de l'intranet
impactent de nombreux outils. Il convenait donc d'implémenter une
gestion des droits permettant de limiter l'accés en écriture sur les
bases de données. Je me suis appuyé pour cela sur le framework PHP
interne de Groupama, celui-ci contenait notamment une interface pour
faire le lien entre un formulaire placé sur une page de login et
l'Active Directory (qui pour rappel est un service d'identification 
implémenté par Microsoft pour se connecter sur un réseau d'entreprise).

L'idée était donc de permettre à un utilisateur de s'identifier sur
l'intranet, de vérifier son identité dans l'Active Directory, puis de
lui attribué des droits codé "en dur" au niveau de
l'application. L'association entre les utilisateurs et leurs droits
potentiels se fait au niveau d'une petite base de données qui n'était
pas encore accessible depuis l'intranet lui-même avant mon départ.

Si l'utilisateur possède les droits, alors des boutons de mise à jour,
d'ajout et de suppression apparaissent sur les différentes interfaces.
1

\section{Onglets développés}

\subsection{Gestion de serveurs}

j'ai développé cet outil pour faciliter la gestion logistique des
serveurs aux collaborateurs. Il permet via un ensemble de formulaire
d'effectuer des recherches sur les serveurs de la DSI et ainsi d'avoir
les relations entre les serveurs physiques et virtuels. On peut
également via cet outil connaitre l'emplacement des bases de données.

J'ai conçu la base de données associée dont voici le schéma entités-associations:

\begin{figure}[h]
\begin{center}
\includegraphics[scale=0.5]{Images/entites_associations.jpg}
\caption{Schéma entités-associations}
\label{Schéma entités-associations}
\end{center}
\end{figure}

\begin{figure}[h]
\begin{center}
\includegraphics[scale=0.85]{Images/gestion_serveurs.png}
\caption{L'outil gestion de serveurs}
\label{L'outil gestion de serveurs}
\end{center}
\end{figure}



Lorsque la recherche est lancée, l'intranet interroge la base de
données et renvoies les occurences correspondantes dans un
tableau. L'utilisateur peut ainsi choisir le serveur sur lequel il
souhaite avoir des informations puis mettre à jour les informations si
besoin est (et s'il possède les droits bien sur).



Les données sont transmisesd'une page à l'autre via la méthode POST
pour les formulaires et par la méthode GET pour les occurences
présente dans le tableaux.

\subsection{télédistribution}

Cette partie permet aux collaborateurs de la DSI d'installer des
logiciels sur des postes depuis un serveur. Pour le moment, l'outil de
télédistribution utilisé est Radia, et les collaborateurs ont accès à
un certain nombre de statistiques. 

Cette partie est un simple regroupement de liens permettant d'arriver
rapidement aux fonctionnalités de Radia.

\begin{figure}[h]
\begin{center}
\includegraphics[scale =0.85]{Images/teledistribution.png}
\caption{Menu de la partie Télédistribution}
\label{Menu de la partie Télédistribution}
\end{center}
\end{figure}

\subsection{Outil réseau}

Cet outil était déjà présent dans l'ancienne version de l'intranet. Il
fournit de nombreuse informations sur les différents sous-réseaux de
l'entreprise, et notamment toutes les informations de connexion des
différentes agence de la région.

Du fait de son existence dans l'ancienne version, le travail a été
plus rapide que pour l'outil de gestion des serveurs, notamment parce
que les bases de données étaient immédiatement utilisables. j'ai
également réutilisé le même modèle graphique que pour l'outil gestion
des serveurs.

\begin{figure}[h!]
\begin{center}
\includegraphics[scale=0.85]{Images/outil_reseau.png}
\caption{Outil d'information du réseau}
\label{Outil d'information du réseau}
\end{center}
\end{figure}

\subsection{Gestion du référentiel Architecture Technique (RAT)}

Cet outil permet d'avoir des informations sur les différents logiciels
installé au sein de Groupama. Version, nombre de poste sur lesquels
la-dite version est installée, etc.

De même que pour l'outil d'information réseau, ce référentiel éxistait
déjà dans l'ancienne version et j'ai pu récupérer les bases de
données, tout en appliquant le même style graphique que pour les
autres onglets.


\subsection{Sécurité}

Cette partie est un portail vers les différents outils utilisé par 
l'équipe chargé de la sécurité, notamment vers des solutions de sécurité 
comme Symantec ou des statistiques liées au Firewall de Groupama.

\subsection{Téléphonie}

Comme la partie Sécurité, la partie Téléphonie est plus un portails
vers les outils utilisé par l'équipe Téléphonie de Groupama (par
exemple vers la suite Genesys d'Alcatel-Lucent).




\section{Environnement logiciel final}

L'intranet, en plus d'être un portail vers des outils externes à la
DSI (par exemple vers HP Network Node manager qui sert à gerer le
reseau), sert donc également à accéder à des outils interne à la DSI
(notamment à Easy Vista qui permet entre autre d'associer du materiel
à un collaborateur ou au référentiel collaborateur) et d'interface
pour alimenter plusieurs bases de données auxquelles accèdent ces
outils.

D'un point du vue de la réalisation, voici comment se situe l'intranet après la refonte:

\begin{figure}[h!]
\begin{center}
\includegraphics[scale=0.85]{Images/Environnement_logiciel.png}
\caption{Environnement Logiciel}
\label{Environnement Logiciel}
\end{center}
\end{figure}
 
