\section{Choix Technique}
\thispagestyle{fancy}

\subsection{Structure du code}
Les différentes classes composants notre code:
\begin{itemize}
\item tutorials est une classe que nous avons créé et qui contiens les
  fonctions de mogre permettant notamment d'initialiser les caméras,
  le clavier...
\item environnement est la classe qui gère notre monde, elle instancié
  le terrain, les pierres, les premières entités, les lumières...
\item character est une classe abstraite qui sert de base pour nos
  agents.
\item ninja est évidemment la classe gérant les ninjas, elle hérite de
  la classe character, elle est autonome, a l'exception des demandes
  de position qu'elle effectue à l'environnement.
\item robot hérite également de character et fait la même chose que ninja mais pour les robots.
\item stone est la classe qui permet de gérer les pierres. 
\end{itemize}

\subsection{principaux attributs des agents}

Les agents possèdent plusieurs attributs modifiables rapidement, Les
deux suivants permettent de configurer les simulations:

\begin{description}
\item[Le champs de vision :] Le champs de visions peut-être modifié
  facilement grâce a deux attributs, maxView et viewinAngle qui gèrent
  respectivement la distance maximale d'un point pouvant être vu par
  l'agent et l'angle entre le vecteur de la direction et la limite du
  cone de son champs de vision.
\item[vitesse :] On peut modifier la vitesse des personnage de deux
  façons. Si on veut accelerer la vitesse du monde, on modifie
  l'attribut de character mWalSkpeed qui agit sur les classe
  d'agent. Si l'on souhaite modifier la vitesse d'un agent par rapport
  à l'autre, il suffit de modifier l'attribut walkSpeedFactor d'un
  agent.
\end{description}
\subsection{Environnement}
L'objet environnement stock toutes les informations de la simulation. Il sert d'interface entre les différent agent. Pour trouver des pierres les agent interrogent "environnement" qui en réponse leur envoie une liste de pierre qu'ils perçoivent en fonction de leurs angle et distance de vue. Environnement contient une méthode lookCharacter qui permet de trouver les personnage que voie. Dans une première version les robot suivait les ogres et devait leur voler leur pierre, mais nous n'avons pas obtenu le résultat souhaité nous avons donc changer d'idée sur le comportement des robot.
Cette objet sert également à gérer l'ajout ou la suppression de personnage, ainsi que les paramètre initiaux de la simulation tels que la taille du terrain, les lumières, la distribution des pierres. 
\subsection{Aléatoire}
Les mouvements aléatoire sont produits en générant deux nombres
aléatoires, qui sont limité pour que l'agent ne puissent pas recevoir
de position à atteindre qui soit hors du terrain.  On utilise
également des nombres aléatoires pour les positions de départs des
agents.

\subsection{limites du terrain}

Une méthode "outOfGround" permet à un agent d'interroger
l'environnement pour savoir s'il va sortir du terrain, ce qui n'est
pas sensé arrivé. Dans le cas contraire, il est renvoyé vers une autre
positions a l'interieurs du terrain.

\subsection{environnement}
La classe environnement initialise le monde et sais tout. Elle a la liste des 
agents, des pierres, connait leur positions. c'est cette classe qui determine à quelles informations sur le monde les agents ont accès.
