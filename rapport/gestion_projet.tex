\section{Gestion de projet}
\addcontentsline{toc}{chapter}{Gestion de projet}
\thispagestyle{fancy}
Pour ce projet nous avons choisi un développement de Type agile. A chaque début de séance nous nous fixons un objectif à atteindre, la séance s'arrête lorsque l'objectif est atteint ou lorsque. Si l'objectif est atteint et qu'il nous reste du temps, un nouvelle objectif est déterminé. Les séance de travail avait lieur sur les heures de projet qui nous était consacré le lundi matin ainsi que sur de nombreux week-end. Pour avoir une vision à plus long terme du projet nous avons fixé des objectif généraux avec une deadline. 

\noindent Les objectifs choisis était :
\begin{itemize}
\item le 24 Novembre 2013 la partie moteur 3D (déplacement de caméra, lumière personnage qui marche suivant une lois déterminé)devait être fini.
\item le 8 Décembre 2013 le comportement des ninjas devait être terminé
\item le 18 Décembre 2013 le comportement des robots devait être terminé
\item le 20 Décembre 2013 les simulation devait être terminé ainsi que le rapport
\end{itemize}

Globalement les deadline ont été respecté, sauf pour la partie moteur 3D. Le moteur 3D nous à pris beaucoup plus de temps que nous l'estimons car le documentation mogre n'est malheureusement pas toujours à jour est difficilement trouvable, en particulier comment était coder le tutorialFramework montrer dans "mogre basic tutorials". Finalement nous avons réussi a trouver le dépôt git de tutorialFramework

  