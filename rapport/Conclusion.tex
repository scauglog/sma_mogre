\chapter{Conclusion}
\thispagestyle{fancy}
Ce projet nous a permis de nous initier aux problématiques d'un environnement 3D. Nous avons due surmonter de nombreuse difficulté due à cette environnement qui nous était peu familier. Nous avons fait le choix de réécrire la partie initialisation de la simulation et de ne pas utiliser une libraire qui nous était opaque, ainsi nous avons pu comprendre le fonctionnement de mogre ce qui nous a faciliter nos développement lors de l'implémentation du système multi-agent. Malheureusement nous avons perdu beaucoup de temps sur cette partie et nous avons pas pu implémenter toute les comportement que nous avions prévue pour nos agent. Ainsi nous avions penser leur donner une capacité de reproduction avec des caractère héréditaire (méthode activer ou non en fonction de celle des parents).

Même si nous n'avons pas implémenter tout ce que nous souhaitions pour nos agents, nous avons réussi a leur implémenter un comportement simple qui lorsqu'il sont plusieurs fait émerger un comportement de groupe, ce phénomène est surtout visible pour les ninjas.