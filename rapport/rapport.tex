\documentclass[a4paper,11pt,final]{report}
% Pour une impression recto verso, utilisez plutôt ce documentclass :
%\documentclass[a4paper,twoside,11pt,final]{article}

\usepackage[francais]{babel}
\usepackage[utf8]{inputenc}
\usepackage[T1]{fontenc}
\usepackage[pdftex]{graphicx}
\usepackage{setspace}
\usepackage[colorlinks=true,linkcolor=black,urlcolor=blue]{hyperref}
\usepackage[french]{varioref}
\usepackage[top=7em, bottom=7em, left=7em, right=7em]{geometry}
\usepackage{wrapfig}
\usepackage{fancyhdr}
\fancyhf{}
\renewcommand{\footrulewidth}{0.05em}
\lhead{\leftmark}
\cfoot{\thepage}
\lfoot{\reportauthor}
\rfoot{Projet SMA}

\newcommand{\reporttitle}{Conception d'un système multi-agent dans un environnement 3D}     % Titre
\newcommand{\reportauthor}{Louis \textsc{LE CLEC'H}\\
                           Romain \textsc{SAGEAN}} % Auteur
\newcommand{\reportsubject}{} % Sujet
\newcommand{\HRule}{\rule{\linewidth}{0.5mm}}
\newcommand{\siecle}[1]{\textsc{\romannumeral #1}ème}
\setlength{\parskip}{1ex} % Espace entre les paragraphes


\hypersetup{
    pdftitle={\reporttitle},%
    pdfauthor={\reportauthor},%
    pdfsubject={\reportsubject},%
   % pdfkeywords={rapport} {vos} {mots} {clés}
}

\begin{document}
  \begin{titlepage}
\begin{flushleft} 
  \large
  \emph{Auteur :}\\
  \reportauthor\\
\end{flushleft}
\begin{center}
  
  \begin{center}
    \begin{tabular}{c}
      \\
      \includegraphics [width=55mm]{Images/ENSEIRB.png} \\
      \\
      \includegraphics [width=50mm]{Images/ENSC.jpg}\\
      \\
      \\
    \end{tabular}
    
    
      
    \textsc{\Large \reportsubject}\\[0.5cm]
           {\large 20 Decembre 2013}\\
           
           
           \HRule \\[0.4cm]
                  {\huge \bfseries \reporttitle}\\[0.4cm]
                  \HRule \\[1.5cm]
                  
                  \begin{center}
                    
                    
                    
                    
                  
                  
                  %\vfill                  
                  
                  
                  
  \end{center}    
\end{center}
\begin{flushbottom}
\begin{flushright} 
  \large
  \emph{Responsable :} \\
  M.~Pierre Alexandre Favier\\

\end{flushright}
\end{flushbottom}
\end{center}
\end{titlepage}

  %\cleardoublepage % Dans le cas du recto verso, ajoute une page blanche si besoin
  \tableofcontents % Table des matières
  \listoffigures
 % \sloppy          % Justification moins stricte : des mots ne dépasseront pas des paragraphes
  %\cleardoublepage
  \begin{onehalfspace}
  \pagestyle{fancy}
  
  \chapter{Introduction}
\thispagestyle{fancy}
Le but de ce projet est de nous familiariser avec les environnement 3D et de faire une première approche des système multi-agent.

 



  
  
  \chapter{Conclusion}
\thispagestyle{fancy}
Ce projet nous a permis de nous initier aux problématiques d'un environnement 3D. Nous avons due surmonter de nombreuse difficulté due à cette environnement qui nous était peu familier. Nous avons fait le choix de réécrire la partie initialisation de la simulation et de ne pas utiliser une libraire qui nous était opaque, ainsi nous avons pu comprendre le fonctionnement de mogre ce qui nous a faciliter nos développement lors de l'implémentation du système multi-agent. Malheureusement nous avons perdu beaucoup de temps sur cette partie et nous avons pas pu implémenter toute les comportement que nous avions prévue pour nos agent. Ainsi nous avions penser leur donner une capacité de reproduction avec des caractère héréditaire (méthode activer ou non en fonction de celle des parents).

Même si nous n'avons pas implémenter tout ce que nous souhaitions pour nos agents, nous avons réussi a leur implémenter un comportement simple qui lorsqu'il sont plusieurs fait émerger un comportement de groupe, ce phénomène est surtout visible pour les ninjas.
  \begin{spacing}{2.0}
    \include{Bibliographie}
  \end{spacing}
  \include{Annexe}
\end{onehalfspace}
\end{document}
