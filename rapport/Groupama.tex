\chapter{Presentation de groupama}

Groupama est une mutuelle d'assurance française, et une banque généraliste. Elle est présente dans 14 pays et s'appuie en france sur trois marques, Groupama, Gan, et Amaguiz.com.

\section{Quelques chiffres}

\begin{itemize}
\item 8ème assureur français, (classement FFSA-GEMA, juin 2012).
\item 13 millions de sociétaires et clients.
\item 35 000 salariés.
\item 6,3 milliards d’euros de fonds propres (Périmètre Groupe).
\item 14,2 milliards d’euros de chiffre d’affaires.
\item Métiers en France : 80\% du CA du groupe:
  \begin{description}
  \item [1er assureur :] agricole, collectivités locales, santé individuelle, protection juridique.
  \item [2ème assureur:] automobile et assureur prévoyance individuelle.
  \item [3ème assureur:] habitation et des professionnels.
  \end{description}
\item Présence internationale : dans 11 pays
\item Activités financières : banque et sociétés de gestion ; 90,4 milliards d’euros gérés par Groupama Asset Management.
\end{itemize}

\section{Histoire}

Les assurances mutuelles agricoles (AMA) sont nées dinitiatives locales d'agriculteurs, pour répondre aux aléas de leurs métier et protéger leurs exploitations et leurs familles contre les risques et les aléas inhérents à leurs activités. Elles ont commencé a se former au \siecle{19} siècle, mais ont réellement pris une existence officielle grâce a la loi du juillet qui les encadres juridiquement. Les AMA permettaient aux agriculteurs de s e prémunir des risques professionnels comme les incendies, la mortalité des bétails, etc.

\begin{itemize}
\item En 1953, le risque agricole est étendu et on se raproche des assurances que l'on connait aujourd'hui. Les AMA permettent d'assurer également les bien et les risques non-professionels des exploitants et de leur famille, y compris le risque automobile. 

\item Pendant les années 60 et 70 les AMA, via plusieurs filiale (Samda, Soravie, Sorema), les AMA diversifient encore leurs activités: converture des risques et dommages non-agricole, "bancassurance" grâce à un partenariat avec le crédit agricole, lancement à l'international...

\item En 1986, les AMA et leurs filiales se regroupent sous la marque Groupama, et l'on va assister à une concentration des caisses régionales. De 70 à l'origine, elles ne sont plus que 9 aujourd'hui (à verifier). Ainsi Groupama Mutasudest, Groupama Loire/Haute Loire, et Groupama des savoies ont donné naissance à Groupama Rhône-Alpes en 1991.

\item En 1995, la société change radicalement son mode de fonctionnement puisque les assurés non agricole deviennent des sociétaires à part entière de leur mutuelle. Groupama Rhône-Alpes fusionne avec Groupama est-centrale, puis en 2003 avec Groupama centre-sud, pour former la structure actuelle, Groupama Rhône-Alpes Auvergne.

\item En 1998 Groupama rachète GAN, qui lui permet de s'implenter en milieu urbain et de rester competitif sur un marché de plus en plus saturé. Groupama devient avec cette acquisition le deuxième assureur généraliste français, et le dixième européen.

\item Groupama devient en 2001 le premier actionnaire du groupe Scor (un réassureur français).

\item En 2003 Groupama se lance dans l'activité bancaire grâce à sa filiale Groupama Banque, qui fusionneras en 2009 avec Banque Finama. 
\end{itemize}

Ces dernières années, Groupama a continué a regrouper ces agences régionales et à revendus plusieur filiales étrangère (en grèce par exemple).

\section{Organisation du groupe}

\begin{figure}[h]
\begin{center}
\includegraphics[scale=0.45]{Images/carte-france-caisses-regionales.jpg}
\caption{La carte des différentes caisses régionales}
\label{La carte des différentes caisses régionales}
\end{center}
\end{figure}

Au total, 9 caisses régionales métropolitaines au 1er janvier 2012, 2
caisses spécialisées : Sylviculteurs (MISSO) et Producteurs de tabac ;
2 caisses d’outre-mer (Antilles-Guyane, Océan Indien). Les caisses
régionales sont des entreprises de plein exercice.  Assureurs
généralistes, elles disposent, notamment de réseaux commerciaux de
salariés et de services de gestion.  Au niveau régional, les élus se
prononcent sur les orientations générales et sur la politique
commerciale de la caisse régionale.  La gestion opérationnelle est
déléguée au Directeur général de la caisse.

\begin{figure}[h]
\begin{center}
\includegraphics[scale=0.45]{Images/schema-organisation.png}
\caption{Organisation de Groupama}
\label{Organisation de Groupama}
\end{center}
\end{figure}

Les caisses régionales sont vraiment l'épine dorsale du groupe car ce
sont elles qui composent la holding Groupama S.A..Elles sont composée
de 3600 caisses locales qui sont représentées par 50000
administrateurs élus par les sociétaires.

\section{Groupama Rhône-Alpes Auvergne}

J'ai éffectué mon stage à la caisse régionale Groupama Rhône-alpes
Auvergne qui se trouve à Lyon.  Cette caisse régionale est composée de
12 fédérations départementales, de 480 caisses locales et de 7300
administrateurs, ce qui en fait une des plus grosse caisse régionale
de France. Elle possède 580000 client et génère a elle seule 971,6
millions d'euros de chiffre d'affaire.  Elle emploie 2000 salariés
répartie en 9 sites de gestion structuré en pôle de competence.

J'ai travaillé au sein de la DSI de cette caisse régionale qui est
chargé de developper les projets informatiques au niveau local, et de
relayer les grands projets nationaux.  La DSI gère également le réseau
de Groupama sur la région, et a un rôle d'assistance technique chargé
d'aider et de former les collaborateurs sur les questions techniques.
Enfin elle a pour mission d'assurer le deploiement des flottes de
terminaux fournis aux collaborateurs.
 
\begin{figure}[h]
\begin{center}
\includegraphics[scale=0.55]{Images/siege_groupama.jpg}
\caption{Photo du siège de la caisse régionale de Groupama Rhône-alpes Auvergne}
\label{Photo du siège de la caisse régionale de Groupama Rhône-alpes Auvergne}
\end{center}
\end{figure} 


